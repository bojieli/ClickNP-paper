%add something here
\documentclass[10pt,pdfletax,letterpaper]{sig-alternate-05-2015}

\usepackage{cite}
\usepackage{times}
\usepackage{amsmath}
\usepackage{graphicx}
\usepackage{graphics}
\usepackage{epsfig}
\usepackage{epstopdf}
\usepackage{latexsym}
\usepackage{amsfonts}
\usepackage{amssymb}
\usepackage{paralist}
\usepackage{xspace}
\usepackage{mathrsfs}
\usepackage{amssymb}
\usepackage{color}
\usepackage{algorithm}
\usepackage{algorithmic}
\usepackage{listings}
\usepackage{multirow}
\usepackage{booktabs}
\usepackage{tabularx}
\usepackage{subfigure}
\usepackage{url}
\usepackage{courier}
\usepackage{balance}

\lstset{basicstyle=\footnotesize\ttfamily,breaklines=true}
\lstset{framextopmargin=50pt}
\lstset{numbers=left}
\graphicspath{{image/}}

\setlength{\pdfpagewidth}{8.5in}
\setlength{\pdfpageheight}{11in}
%% SIGCOMM guidelines:
% Columns are 9.25 tall
% 3.33" wide with 0.33" separation
% -> 3.33"*2+0.33 = 7" -> 0.75 per left and right
% 0.75" top and 1" bottom should work out for 9.25" tall
% footskip is where the page #s go.
\usepackage[letterpaper,nohead,
	left=0.75in,right=0.75in,top=0.75in,
	footskip=0.5in,bottom=1in,
	columnsep=0.33in]{geometry}

\begin{document}

\CopyrightYear{2016}
\setcopyright{acmcopyright}
\conferenceinfo{SIGCOMM '16,}{August 22--26, 2016, Florianopolis, Brazil}
\isbn{978-1-4503-4193-6/16/08}
\acmPrice{\$15.00}
\doi{http://dx.doi.org/10.1145/2934872.2934897}


% What is going to be the name
%\title{ClickNP: A Modular Software Network Processor on Reconfigurable Hardware \vspace{-.5cm}}
\title{ClickNP: Highly Flexible and High Performance Network Processing with Reconfigurable Hardware\vspace{-0.8cm}}
\def\name{ClickNP}
\def\fullname{Click Network Processor}
\def\sysname{}

%% \vspace*{-0.1in}

\newcommand{\authornote}[1]{\raisebox{0.8ex}{$#1$}}

\numberofauthors{1}
\author{
	Bojie Li\authornote{\S\dagger} \and
	Kun Tan\authornote{\dagger} \and
	Layong (Larry) Luo\authornote{\ddagger} \and
	Yanqing Peng\authornote{\bullet\dagger} \and
	Renqian Luo\authornote{\S\dagger} \and
	Ningyi Xu\authornote{\dagger} \and
	Yongqiang Xiong\authornote{\dagger} \and
	Peng Cheng\authornote{\dagger} \and
	Enhong Chen\authornote{\S} \and
	\authornote{\dagger}Microsoft Research \quad
	\authornote{\S}USTC \quad
	\authornote{\ddagger}Microsoft \quad
	\authornote{\bullet}SJTU
	%\scalebox{0.8}{\{v-bojli, kuntan, laluo, v-reluo, v-yanpen, ningyixu, yqx, pengc\}@microsoft.com, cheneh@ustc.edu.cn}
}

\maketitle

%
% The code below should be generated by the tool at
% http://dl.acm.org/ccs.cfm
% Please copy and paste the code instead of the example below. 
%
\begin{CCSXML}
	<ccs2012>
	<concept>
	<concept_id>10003033.10003058.10003063</concept_id>
	<concept_desc>Networks~Middle boxes / network appliances</concept_desc>
	<concept_significance>500</concept_significance>
	</concept>
	<concept>
	<concept_id>10003033.10003106.10003110</concept_id>
	<concept_desc>Networks~Data center networks</concept_desc>
	<concept_significance>300</concept_significance>
	</concept>
	<concept>
	<concept_id>10010583.10010682.10010684.10010686</concept_id>
	<concept_desc>Hardware~Hardware-software codesign</concept_desc>
	<concept_significance>300</concept_significance>
	</concept>
	</ccs2012>
\end{CCSXML}

\ccsdesc[500]{Networks~Middle boxes / network appliances}
\ccsdesc[300]{Networks~Data center networks}
\ccsdesc[300]{Hardware~Hardware-software codesign}


%
% End generated code
%

%
% import the customized commands
%
\input{kuncustom}
\input{abstract}

%
%  Use this command to print the description
%
\printccsdesc

% We no longer use \terms command
%\terms{Theory}
\vspace{-5pt}
\keywords{Network Function Virtualization; Compiler; Reconfigurable Hardware; FPGA}


%
% section by section tex files
%
\input{intro}
\input{background}
\input{architecture}
\input{optimize}
\input{impl}
\input{application}
%\input{language}
%\input{elements}
\input{eval}
\input{related}
%\input{future}
\input{conclude}
\input{acknowledge}

\balance
\bibliographystyle{abbrv}
{\small \bibliography{reference}}

\end{document}
